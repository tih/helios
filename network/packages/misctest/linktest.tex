\documentstyle{article}
\title{Helios Low-level Link Communications Tester}
\author{Bart Veer \\ PSL-HEL-BLV-93-003.2}
\begin{document}
\maketitle

The purpose of this test program is to perform some testing of the
low-level communication routines provided by the kernel,
{\tt LinkIn()} and {\tt LinkOut()}, and thus attempt to validate the
behaviour of the underlying hardware and the link I/O code within
the Helios Executive. This should allow the user to bypass higher
levels of the protocol, in particular the various forms of buffering
performed by the kernel and the flow control mechanisms used within
the pipe code.

\section{Errors Detected}

The program detects five different types of error:

\begin{enumerate}
\item Failures mean data corruption. The data received is not the same as
the data that was sent. The program can send various types of data:
an iteration number, random data with a checksum, and constant data.
\item Read timeouts mean that a call to LinkIn did not complete within
a reasonable time, which is set to some orders of magnitude larger than
the time it should take for the other end to send the next packet. When
a read timeout occurs the operation will be retried a number of times.
\item Fatal read timeouts mean that repeated calls to LinkIn all
failed, in other words all communication with the sending side has been
lost.
\item Write timeouts mean that a call to LinkOut did not complete within
a reasonable time. Again these timeouts will be retried.
\item Fatal write timeouts mean that repeated calls to LinkOut all failed,
in other words it is no longer possible to send data down the link.
\end{enumerate}

\section{Setting up the Network}

The test program is intended to run on a network of processors with
significant link redundancy. For example the network topology known
to Helios might be a simple pipeline, whereas the underlying topology
is much richer. This can be achieved simply by not specifying
all the available links in the network resource map.

By default the test program will flood the network and try to use
every link that is not used by Helios. More specifically it will
try to use every link that is currently set to mode 0, null. Hence
the program will not use running links, pending links to external
networks, or dumb links. In addition the program can read in a
file specifying which links should be skipped, to bypass links that are
already known to be faulty. Further details are given below.

\section{Running the Test Harness}

The {\tt linktest} program takes a number of arguments. It must be given
the duration of the run, some number of days, hours and
minutes. For example to run the program quickly for a couple of minutes
use the command:

\begin{verbatim}
   linktest 2
\end{verbatim}

To run the program overnight use the command:

\begin{verbatim}
   linktest 14 0
\end{verbatim}

To run the program over an entire weekend use the command:

\begin{verbatim}
   linktest 2 14 0
\end{verbatim}

In addition the program takes a number of optional arguments, as follows:

\begin{description}
\item [{\tt -v}] verbose mode, this causes the program to report at
various points during the run and in particular whenever a failure is
detected.
\item [{\tt -V}] really verbose mode. This enables all the output
produced by verbose mode plus quite a lot of extra information.
\item [{\tt -1}] run in uni-directional mode. By default the program will
flood a link with traffic in both directions. In uni-directional
mode the two processors at the end of a link will compare their processor
names using {\tt strcmp()} and the alphabetically smaller processor
will transmit, with the other receiving. If two links of the same
processor are connected then the lower-numbered link will be used to
transmit and the higher-numbered link will receive.
\item [{\tt -r}] by default the program transmits fairly constant data across
the links: the first packet will contain all zeros, the second packet
all ones, and so on. As an alternative the program can transmit blocks
of random data with the first word being a checksum of the remainder.
\item [{\tt -k}] this forces the program to transmit a single constant number
only instead of an iteration number.
\end{description}

\section{The Skip File}

Under certain conditions it may be desirable to force the program to
ignore particular links. For example a link may be connected to an
M212 disk controller, or a particular link may be known to be faulty.
When the program starts up it consults a {\em skip file} which lists
all the links that should be ignored. By default this will be the file
{\tt linkskip} in the current directory, but an alternative file can
be specified using the environment variable \verb+LINKTEST_SKIPFILE+.

A skip file should consist of one or more lines, each line giving a
processor name and a link number. For example:

\begin{verbatim}
  01 0
  02 3
\end{verbatim}

This would force the program to ignore link 0 on processor {\tt 01} and link 3
on processor {\tt 02}. Note that the processor names are not absolute
pathnames.

\section{The Logfile}

The program maintains a logfile of all runs. This contains some summary
information at the start of the file as well as information from every
individual run. The default logfile will be {\tt linktest.log} in the
current directory, but an alternative logfile can be specified using
the environment variable \verb+LINKTEST_LOGFILE+.

\end{document}

