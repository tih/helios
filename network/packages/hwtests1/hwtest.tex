\documentstyle{article}
\title{Helios 1.3 Hardware Reliability Test Program}
\author{Bart Veer \\ PSL-HEL-BLV-92-009.1}
\date{}

\begin{document}
\maketitle

This simple hardware test program was developed as part of the Helios
fault tolerance work, in an attempt to measure the long-term reliability
of existing hardware. The aim is to detect failures within some of the
various components that make up a processor node in a multiprocessor network.
For example the test program can detect corruption of single bits of
memory caused by alpha radiation, cosmic particles, etc.

The test program is not intended to identify failures in existing
hardware, for example a failure in the floating point unit which means
that one bit in the result is always set. Instead it assumes that the
hardware platform being used is reliable but subject to sporadic failures,
once every few weeks or months depending on the size of the network.
The frequency of such sporadic failures is not really known and is
clearly rather hard to measure. Failures are assumed to be soft, in other
words one-off events such as an alpha particle zapping a bit in memory,
rather than hard failures such as the permanent failure of one bit in
memory or an entire memory chip.

In addition to hardware testing the program can be used to check
the behaviour of Helios. In particular it serves to test the taskforce
mapping and execution facilities of the Resource Management library 
and the long-term stability of the nucleus, as the program is typically
run for hours or days on end.

\section{The Testing}

When the program is run it examines the current network, obtains all
suitable processors, and then constructs a taskforce to run on all
the available processors. There will be one component, internally
multi-threaded, for every processor. Each such component performs
some or all of the following tests:

\begin{description}
\item [links] For every link on this processor which is connected to another
processor owned by the test program, there will be a pair of communication
channels to the program on the other end. The test program will continually
transmit data on one channel and receive it on the other. Packets start
with eight bytes of random data and double, going up to 16K. The first word of
the transfer consists of a checksum of the remaining data.
\item [on-chip memory] The program determines the amount of on-chip
memory that is available and obtains it. This memory is initialised with
random data and then checksummed at regular intervals.
\item [off-chip memory] Similarly the program determines the amount of
off-chip memory, in other words SRAM and DRAM. Again it determines the
amount of memory available, making certain that the link testing
threads have allocated their buffers first. Care is taken that the system
is left with at least 50K of memory for message buffering and so on.
The obtained memory is again filled with random data and checksummed
at regular intervals.
\end{description}

Only a single program is used to perform all the testing. This program
is identical to the {\tt hwtest} program itself, but given suitable
arguments.

In addition to the above checking the program performs some testing
of Helios. In particular it keeps track of the following:

\begin{enumerate}
\item The size of the port table. If this grows by more than two then
a warning will be reported immediately.
\item The size of the free pool. If this drops below 25K a warning will
be generated.
\item The trace vector. If the kernel generates any traces then these
will be reported.
\item The kernel buffer count. If this exceeds the maximum count currently
seen then a warning will be generated.
\end{enumerate}

The aim of these tests is to detect internal problems in Helios,
particularly ones related to long-term stability.

\section{Logging}

The test program generates a logfile. By default this is \newline
{\tt /helios/local/tests/hwtests1/hwtest.log}, but an alternative logfile
can be specified using the environment variable \verb+HWTEST_LOGFILE+.
The start of this logfile consists of summary information:

\begin{enumerate}
\item Total amount of link traffic transferred, in gigabytes, megabytes,
and kilobytes.
\item Number of link failures detected.
\item Total amount of on-chip memory that has been tested. These
unit are on a per-hour basis, in other words the number of gigabytes
tested for an entire hour, then the number of megabytes tested for
an entire hour, finally the number of kilobytes.
\item Number of on-chip memory errors detected.
\item Total amount of off-chip memory that has been test, again in
gigabyte hours, megabyte hours, and kilobyte hours.
\item Number of off-chip memory errors detected.
\end{enumerate}

This summary information will be followed by a log of the actual
testing performed, in chronological order. Every time the test is run
some data will be appended to the end of the log.

\section{Command line Arguments}

The {\tt hwtest} program takes the following arguments:

\begin{description}
\item [{\tt -v}] Verbose. In addition to the normal logging information
sent to the logfile the program will output detailed results from all
the components. This information will be sent to {\tt stderr}.
\item [{\tt -d}] Detach. The taskforce will be run in the background,
provided that the user's session remains. If the user logs out then
the taskforce will be aborted automatically.
\item [{\tt -m}] Memory interval. This option should be followed by
a time interval in seconds. By default the test program performs a
complete checksum of all on-chip and off-chip memory every ten seconds.
An alternative interval can be specified if desired. For example an
interval of 0 forces continuous memory checking, whereas a larger
interval allows for more link traffic to be generated and checksummed.
\item [{\tt -nl}] No link checking. This option disables the link
checking. 
\item [{\tt -nm}] No memory checking. This option disables the off-chip
memory checking.
\item [{\tt -nf}] No fast memory checking. This option disables the
testing of on-chip memory.
\item [{\tt -c}] Perform network communication testing as well as
the hardware testing. If this option is enabled then the controller
generates large quantities of communication in addition to the
communication across the links. The aim is to stress-test the
message passing code, particularly single and double buffering,
under low memory and high communication activity conditions.
\item [time] The user should specify the duration of the test run.
If three arguments are given then this is interpreted as days, hours
and minutes. If only two arguments are given then this is interpreted
as hours and minutes. If only one argument is given then the program
only runs for the specified number of minutes.
\end{description}

For example to run the test overnight the following command line could
be used:

\begin{verbatim}
  hwtest 14 0
\end{verbatim}

To perform link testing only over a weekend, generating verbose output,
the following command line could be used:

\begin{verbatim}
  hwtest -v -nf -nm 2 15 0
\end{verbatim}

\section{Internal Arguments}

When run from the command line the {\tt hwtest} program examines the
network, obtains all suitable processors, and creates a taskforce.
This taskforce consists of a number of copies of the {\tt hwtest}
program, with the id \verb+<hwtest>+ to allow these copies to determine
that they are running as part of the taskforce. These taskforce components
take the following arguments:

\begin{description}
\item [days] always present, 0 if the user did not specify otherwise.
\item [hours] ditto.
\item [minutes] ditto.
\item [{\tt -l0.4}] communicate with another component running on
the processor attached to link 0. Use channel 4 to receive data from
that component and channel 5 to send data to that component. If the
{\tt -nl} option was specified by the user then there will not be any
of these arguments.
\item [{\tt -nm}] do not check off-chip memory, as per the command line
argument.
\item [{\tt -nf}] do not check on-chip memory, as per the command
line argument.
\item [{\tt -m20}] the interval between successive memory checks in
seconds, as per the command line argument.
\item [{\tt -c8.20}] Run as the controller, responsible for collecting
the results at the end of the run and updating the log file. In this
case the controller interacts with a further eight components, in other
words the test is running on nine processors. Channel 20 is used to
receive data from the first worker, channel 21 is used to send data to the
first worker, and so on. Each worker that is not running as a controller
uses {\tt stdin} to read from the controller and {\tt stdout} to write
to the controller.
\item [{\tt -C}] Perform communication stress testing.
\end{description}

\section{Possible Enhancements}

There should be another part of the test program responsible for the
floating point unit of a processor. This takes up a significant amount
of the silicon area on most chips and should be tested. The rest of the
processor chip, for example instruction pipelines, ALUs, caches, are
exercised simply by running Helios and the test program, though not tested
exhaustively. However the floating point unit is rarely exercised by
Helios.

Some work may be needed to cope with unusual memory layouts where the
various blocks of memory are not necessarily contiguous. Currently the
program will only allocate and checksum the single largest block of
free memory, not a collection of blocks, which causes problems when
memory is fragmented.

\end{document}


