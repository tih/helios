\documentstyle{article}
\title{Helios Taskforce Test-suite}
\author{Bart Veer \\ PSL-HEL-BLV-92-002.2}

% If you want to produce an on-line version of this document enable
% the following option, process with LaTeX as usual, and then use
% dvi2tty on the output. The result should be 80 column text. It will not
% look perfect, but it will do.
%\setlength{\textwidth}{30em}

\begin{document}
\maketitle

The purpose of the Taskforce test suite is to run a fairly large number of
taskforces through the CDL compiler and the Taskforce Manager and ensure that
they all run correctly. The aim is to provide a fairly quick way
of checking that changes to the Taskforce Manager have not caused any
problems.

This test suite does not attempt to be an exhaustive test of the CDL
compiler. It does not make use of all the CDL syntax, nor does it check
error handling or anything else. However the CDL scripts passed through
the compiler during a run of this test suite should be fairly
representative of actual user applications.

Similarly the test suite does not exhaustively test the Taskforce Manager.
For a start, it does not test any of the Resource Management library
interface since that is the responsibility of a separate test suite.
It does not test exception handling such as crashed processors, and only
performs a very basic check on the mapping algorithms.

Most importantly, the test suite does not perform any exhaustive tests
of the communication between the components. Since intensive
communication has been a major reliability problem in the past such
testing is rather desirable and may be added in the future. Given the
fairly large number of topologies executed by this test suite,
it would be a good place to perform communication testing. Unfortunately
it would add considerably to the time taken to run the test suite.
For the same reason, the test suite does not currently check that
taskforces can run for more than a few seconds.

\section{Running The Tests}

This test suite should be installed in {\tt /helios/local/tests/tftests}.
The user will require write-access to this directory when running the
test suite, since some of the tests involve file I/O.

To run the test harness just use the {\tt tftests} command. This takes
one optional argument, {\tt -v}, to enable verbose mode. By default
the test suite should run to completion without producing any output.
The verbose mode causes the {\tt tftests} program to report the current
phase of the testing. When an error is detected it may be produced
by the {\tt tftests} program itself, by the CDL compiler, or by a
taskforce component.

The test harness should work in a normal Helios world, preferably with at
least 8 user processors. Since it always fills as much of the network as
possible it should work fairly happily on much larger networks, say up
to 256 processors. Obviously it will take longer to start taskforces
on such networks, and in the case of grids there may be more taskforces
to try out. The test harness should also work in a Tiny Helios environment.

The {\tt tftests} program performs the following operations.

\begin{enumerate}
\item Check that there is a Taskforce Manager associated with this program.
Without a TFM it is rather difficult to run any taskforces.
\item Pre-allocate all the processors in the network.
\item Examine the current domain, counting the number of processors.
To avoid problems on networks with limited memory, if the {\tt tftests}
program is running on the same processor as the TFM and this processor
has only a megabyte of memory then it is booked, preventing the CDL
compiler and any workers running on that processor.
\item If there is only one processor in the current domain then I assume
that the current system is running Tiny Helios. Taskforces will be limited
to four components.
\item Execute the various taskforce topologies described below. Since all
the processors have been pre-allocated the TFM is allowed only
a short delay, five seconds, between taskforces for cleaning-up.
\item Release all the processors in the domain.
Now the aim is to test the interaction between
the Taskforce Manager and the Network Server, rather than just the TFM.
\item Run a simple taskforce to force loading of libraries etc.
This taskforce also does nasty things to ram disks, fifos, etc.
\item Wait for a while to allow the TFM to release the processors,
and to allow the Network Server to clean up.
\item Run another taskforce to check up on the cleaners.
\item Run all the taskforce topologies again. The processors will have
to be allocated and released every time, forcing a clean-up, reloading
of libraries, etc. A rather longer delay will be used between taskforce
runs to allow for this: ten seconds plus another one tenth of a second
per taskforce component.
\item The remaining tests cover some of the nastier aspects of life:
\begin{enumerate}
\item Fully mapped taskforces via the PUID facility
\item Partially mapped taskforces
\item Code specifications and \verb+argv[0]+
\item Argument processing, particularly \verb+$1+ and \verb+\$1+
\item Inheritance of the environment strings and standard streams.
\item Termination
\item Signal handling
\end{enumerate}
\end{enumerate}

More tests may be added in the future, as and when problems are identified.

\section{The Topologies}

The test suite attempts to run taskforces in a wide variety of topologies,
believed to reflect accurately the needs of many applications. A number of
them are based on requests by customers who could not figure out a CDL
script for a particular application. At present
the following topologies are supported:

\begin{enumerate}
\item Uni-directional pipeline.
\item Bi-directional pipeline.
\item Uni-directional ring.
\item Bi-directional ring.
\item Ring with separate controller.
\item Chordal ring, with every component connected to its directly-opposite
number. This requires an even number of components.
\item Various other rings with strange and wonderful topologies.
\item Farm without a load balancer.
\item Farm with a load balancer. Note: this does not test the standard
load balancer, instead it uses a private version. Testing the standard
load balancer should be done in another test suite specific to that
utility.
\item Binary tree. 
\item Ternary tree.
\item Two-dimensional toroid grids of various sizes.
With 32 processors I would want to check 2x16, 3x10, 4x8, and 5x6 topologies. 
\item Hypercubes.
\item Fully-connected.
\end{enumerate}

All these taskforces are executed by {\tt popen()}'ing the CDL compiler
and writing a suitable CDL script to the pipe.

\section{The Programs}

The test suite consists of just two programs: {\tt tftests} and
{\tt worker}. The worker program takes at least one argument which defines
the test currently running, plus associated parameters such as grid size.
It uses this argument to determine the correct action to take.
When testing the topologies, this means communicating over all the
channels that have been defined for the current taskforce in a sensible
order, i.e. without causing a deadlock.

\end{document}

