\documentstyle{article}
\title{Vectlib Help Pages}
\author{Bart Veer \\ PSL-HEL-BLV-92-008.2}
\date{}
\newcommand{\helpentry}[1]{\newpage\section{#1}}

\begin{document}
\maketitle

This document contains a number of help pages which should be incorporated
into the on-line help database. Some of the information may also be used for
the Helios 1.3 introductory notes.

%%{{{  Vector Library
\helpentry{Vector Library (General interest)}

Scientific and engineering calculations often involve performing the same
operation on one or two dimensional arrays of floating point numbers,
known as vectors and matrices respectively. A typical example of such
a calculation is matrix multiplication.

\begin{verbatim}
    double A[256][256], B[256][256], C[256][256];
    int i, j, k;

    for (i = 0; i < 256; i++)
     for (j = 0; j < 256; j++)
      { C[i][j] = 0.0;
        for (k = 0; k < 256; k++)
         C[i][j] += A[i][k] * B[k][j];
      }
\end{verbatim}

In this case the inner loop is executed more than 16 million times. A compiler
can be expected to produce fairly good code for such loops but not
necessarily optimal code. Hand-written assembler code might cut one or
two instructions from the inner loop, with a corresponding improvement in
performance.

The purpose of a vector library is to provide near-optimal or optimal
implementations for common inner loops, at the cost of a procedure call.
For example the above matrix multiplication could be implemented as:

\begin{verbatim}
    for (i = 0; i < 256; i++)
     for (j = 0; j < 256; j++)
      C[i][j] = VdsDot(256, &(A[i][0]), 1, &(B[0][j]), 256);
\end{verbatim}

The vector library supplied with Helios is somewhat limited. Only a fairly
small number of routines are provided. The performance is generally
better than that of code produced by the compiler but not necessarily
optimal. Much more extensive vector libraries may be available as separate
products, and users should check with their distributor for further
information.

The vectors passed to the routines in the vector library must be of at
least 2 elements in length.  This allows the routines to be optimally
coded without regard to the special case of a vector with only one element.
In addition futher improvements in speed can sometimes be obtained by
using the Acclerate() and AccelerateCode() functions.

Only a C interface to the library is defined, although it should be possible
to invoke the library from other languages. The various routines are
declared in the header file \verb+<vectlib.h>+. The following naming
conventions are obeyed.

\begin{enumerate}
\item All routines start with the letter {\tt V}, to reduce name clash problems
with other libraries or user routines.
\item The second letter is {\tt f} for single precision arithmetic, or
{\tt d} for double precision.
\item If the third letter is {\tt s} then this routine involves strides,
in other words the numbers to be operated on are not 
necessarily contiguous in memory.
Typically strides are used for accessing a column of a matrix where the
stride is the number of elements in a row.
\item The final part specifies the operation to be performed, beginning
with a capital letter.
\end{enumerate}

For example the routine {\tt VfsMax()} can be used to find the
largest number in a column of a matrix of single precision numbers.

See also: vector-vector operations, vector-scalar operations,
vector-scalar multiply, vector initialisation, vector copying, vector
dot products, vector sums and products, vector maxima and minima
%%}}}
%%{{{  vector-vector operations
\helpentry{vector-vector operations}

Purpose: to perform a pairwise arithmetic operation on the elements of two
vectors.

Include: \verb+<vectlib.h>+

Format:

\begin{verbatim}
void VfAdd( int n, float * x, float * y );
void VfSub( int n, float * x, float * y );
void VfMul( int n, float * x, float * y );
void VfDiv( int n, float * x, float * y );

void VdAdd( int n, double * x, double * y );
void VdSub( int n, double * x, double * y );
void VdMul( int n, double * x, double * y );
void VdDiv( int n, double * x, double * y );

void VfsAdd( int n, float * x, int x_stride, float * y, int y_stride );
void VfsSub( int n, float * x, int x_stride, float * y, int y_stride );
void VfsMul( int n, float * x, int x_stride, float * y, int y_stride );
void VfsDiv( int n, float * x, int x_stride, float * y, int y_stride );

void VdsAdd( int n, double * x, int x_stride, double * y, int y_stride );
void VdsSub( int n, double * x, int x_stride, double * y, int y_stride );
void VdsMul( int n, double * x, int x_stride, double * y, int y_stride );
void VdsDiv( int n, double * x, int x_stride, double * y, int y_stride );
\end{verbatim}

Argument: {\tt n} -- the number of elements in the vector.

Argument: {\tt x} -- the first vector of floating point numbers.

Argument: \verb+x_stride+ -- the gap between successive elements of
the {\tt x} vector.  This gap is in units of the size of one element
of the vector.  Thus a stride of 1 is equivalent to the non-stride
versions of the functions.

Argument: {\tt y} -- the second vector of floating point numbers.

Argument: \verb+y_stride+ -- the gap between successive elements of the
{\tt y} vector.  Again this is in units of the size of one element of
the vector.

Returns: no value.

Description: these routines perform an arithmetic operation on successive
elements of two vectors, storing the results in the first vector. There
are variants for the four basic operations, the two levels of precision,
and whether or not strides are used. The exact operations implemented are:

\begin{description}
\item [Add] \verb_x[i] = x[i] + y[i]_
\item [Sub] \verb_x[i] = x[i] - y[i]_
\item [Mul] \verb_x[i] = x[i] * y[i]_
\item [Div] \verb_x[i] = x[i] / y[i]_
\end{description}

For example an implementation of {\tt VfSub()} in C might be:

\begin{verbatim}
void VfSub( int n, float * x, float * y )
{
  while (n--)
   *x++ -= *y++;
}
\end{verbatim}

Similarly a C implementation of {\tt VdsMul()} might be:

\begin{verbatim}
void VdsMul( int n, double * x, int x_stride, double * y, int y_stride )
{
  while (n--)
   { *x *= *y; x += x_stride; y += y_stride; }
}
\end{verbatim}

See also: Vector Library, vector-scalar operations, vector-scalar multiply,
vector initialisation, vector copying, vector dot products, vector sums and
products, vector maxima and minima
%%}}}
%%{{{  vector-scalar operations

\helpentry{vector-scalar operations}

Purpose: to perform an arithmetic operation on the elements of a vector
using a single scalar.

Include: \verb+<vectlib.h>+

Format:

\begin{verbatim}
void VfAddScalar( float  value, int n, float * x );
void VfSubScalar( float  value, int n, float * x );
void VfMulScalar( float  value, int n, float * x );
void VfDivScalar( float  value, int n, float * x );
void VfRecScalar( float  value, int n, float * x );

void VdAddScalar( double value, int n, double * x );
void VdSubScalar( double value, int n, double * x );
void VdMulScalar( double value, int n, double * x );
void VdDivScalar( double value, int n, double * x );
void VdRecScalar( double value, int n, double * x );

void VfsAddScalar( float  value, int n, float * x, int stride );
void VfsSubScalar( float  value, int n, float * x, int stride );
void VfsMulScalar( float  value, int n, float * x, int stride );
void VfsDivScalar( float  value, int n, float * x, int stride );
void VfsRecScalar( float  value, int n, float * x, int stride );

void VdsAddScalar( double value, int n, double * x, int stride );
void VdsSubScalar( double value, int n, double * x, int stride );
void VdsMulScalar( double value, int n, double * x, int stride );
void VdsDivScalar( double value, int n, double * x, int stride );
void VdsRecScalar( double value, int n, double * x, int stride );
\end{verbatim}


Argument: {\tt value} -- the scalar used for the operation.

Argument: {\tt n} -- the number of elements in the vector.

Argument: {\tt x} -- the vector of floating point numbers.

Argument: \verb+x_stride+ -- the gap between successive elements of the
{\tt xx} vector.  A stride of 1 is equivelent to the non-stride
versions of the functions.

Returns: no value.

Description: these routines perform an arithmetic operation on
successive elements of a vector using the supplied scalar, storing the
results in the vector. There are variants for the five basic operations,
the two levels of precision, and whether or not strides are used. The
exact operations implemented are:

\begin{description}
\item [Add] \verb_x[i] = x[i] + value_
\item [Sub] \verb_x[i] = x[i] - value_
\item [Mul] \verb_x[i] = x[i] * value_
\item [Div] \verb_x[i] = x[i] / value_
\item [Rec] \verb_x[i] = value / x[i]_
\end{description}

For example an implementation of {\tt VdDiv()} in C might be:

\begin{verbatim}
void VdDiv( double value, int n, double * x )
{
   while (n--)
    *x++ /= value;
}
\end{verbatim}

See also: Vector Library, vector-vector operations, vector-scalar multiply,
vector initialisation, vector copying, vector dot products, vector sums and
products, vector maxima and minima
%%}}}
%%{{{  vector-scalar multiply
\helpentry{vector-scalar multiply}

Purpose: to multiply one vector by a scalar and add the results to another
vector.

Include: \verb+<vectlib.h>+

Format:

\begin{verbatim}
void VfMulAdd(  float  value, int n, float *  x, float * y );
void VfsMulAdd( float  value, int n, float *  x, int x_stride, float *  y, int y_stride );
void VdMulAdd(  double value, int n, double * x, double * y );
void VdsMulAdd( double value, int n, double * x, int x_stride, double * y, int y_stride );
\end{verbatim}

Argument: {\tt value} -- the scalar to be used for the multiplication.

Argument: {\tt n} -- the number of elements in the vector.

Argument: {\tt x} -- the results vector.

Argument: \verb+x_stride+ -- the gap between successive elements in the
{\tt x} vector.  The units are the size of one element of the vector.

Argument: {\tt y} -- the vector to be multiplied.

Argument: \verb+y_stride+ -- the gap between successive elements in the
{\tt y} vector.  The units are the size of one element of the vector.

Returns: no value.

Description: These routines are used to multiple all the elements of
the {\tt y} vector by a scalar value and add the results into the
{\tt x} vector.

\begin{verbatim}
   x[i] = x[i] + (value * y[i])
\end{verbatim}

There are variants for the two levels of precision and for whether
or not strides are used. The {\tt y} vector is left unchanged.
An implementation of {\tt VfMulAdd()} in C might be:

\begin{verbatim}
void VfMulAdd( float value, int n, float * x, float * y )
{
  while (n--)
   *x++ += (value * *y++);
}
\end{verbatim}

See also: Vector Library, vector-vector operations, vector-scalar operations,
vector initialisation, vector copying, vector dot products, vector sums and
products, vector maxima and minima
%%}}}
%%{{{  vector initialisation
\helpentry{vector initialisation}

Purpose: to initialise the elements of a vector.

Include: \verb+<vectlib.h>+

Format:

\begin{verbatim}
void VfFill(  float  value, int n, float *  x );
void VdFill(  double value, int n, double * x );
void VfsFill( float  value, int n, float *  x, int stride );
void VdsFill( double value, int n, double * x, int stride );
\end{verbatim}

Argument: {\tt value} -- the floating point number to be used for initialisation

Argument: {\tt n} -- the number of elements in the vector.

Argument: {\tt x} -- the vector to be initialised.

Argument: {\tt stride} -- the gap between successive elements in the
{\tt x} vector.

Returns: no value.

Description: These routines are used to set all the elements of a vector
of floating point numbers to a particular value. There are variants
for single and double precision arithmetic and for whether or not
strides are used. For example an
implementation of {\tt VfsFill()} in C might be:

\begin{verbatim}
void VfsFill( float value, int n, float * x, int stride )
{
  while (n--)
   { *x = value; x += stride; }
}
\end{verbatim}

See also: Vector Library, vector-vector operations, vector-scalar operations,
vector-scalar multiply, vector copying, vector dot products, vector sums and
products, vector maxima and minima
%%}}}
%%{{{  vector copying
\helpentry{vector copying}

Purpose: to copy a vector of floating point numbers.

Include: \verb+<vectlib.h>+

Format:

\begin{verbatim}
void VfCopy(  int n, float *  x, float *  y );
void VdCopy(  int n, double * x, double * y );
void VfsCopy( int n, float *  x, int x_stride, float *  y, int y_stride );
void VdsCopy( int n, double * x, int x_stride, double * y, int y_stride );
\end{verbatim}

Argument: {\tt n} -- the number of vector elements to copy.

Argument: {\tt x} -- the destination of the copy.

Argument: \verb+x_stride+ -- the gap between successive elements in the
{\tt x} vector.

Argument: {\tt y} -- the source of the copy.

Argument: \verb+y_stride+ -- the gap between successive elements in the
{\tt y} vector.

Returns: no value.

Description: these routines copy {\tt n} floating point numbers from the
vector {\tt y} to the vector {\tt x}. There are variants for the two
levels of precision and for whether or not strides are used. For example
an implementation of {\tt VdsCopy()} in C might be:

\begin{verbatim}
void VdsCopy( int n, double * x, int x_stride, double * y, int y_stride )
{
  while (n--)
   { *x = *y; x += x_stride; y += y_stride; }
} 
\end{verbatim}

See also: Vector Library, vector-vector operations, vector-scalar
operations, vector-scalar multiply, vector initialisation, vector dot
products, vector sums and products, vector maxima and minima
%%}}}
%%{{{  vector dot products

\helpentry{vector dot products}

Purpose: to calculate the dot product of two vectors.

Include: \verb+<vectlib.h>+

Format:

\begin{verbatim}
float  VfDot(  int n, float *  x, float *  y );
double VdDot(  int n, double * x, double * y );
float  VfsDot( int n, float *  x, int x_stride, float *  y, int y_stride );
double VdsDot( int n, double * x, int x_stride, double * y, int y_stride );
\end{verbatim}

Argument: {\tt n} -- the number of elements in each vector.

Argument: {\tt x} -- the first vector of floating point numbers.

Argument: \verb+x_stride+ -- the gap between successive elements of the
{\tt x} vector.

Argument: {\tt y} -- the second vector of floating point numbers.

Argument: \verb+y_stride+ -- the gap between successive elements of
the {\tt y} vector.

Returns: the dot product of the two vectors.

Description: this routine evaluates the dot product of the two vectors.
In other words it calculates the product of successive elements of the
two vectors and returns the sum of these products. The two vectors
remain unchanged.

For example an implementation of the routine {\tt VfDot()} in C might be:

\begin{verbatim}
float VfDot( int n, float * x, float * y )
{ float result = 0.0;

  while (n--)
   result += (*x++ * *y++);
 
  return(result);
}
\end{verbatim}

See also: Vector Library, vector-vector operations, vector-scalar operations,
vector-scalar multiply, vector initialisation, vector copying, vector sums and
products, vector maxima and minima
%%}}}
%%{{{  vector sums and products
\helpentry{vector sums and products}

Purpose: to calculate the sum or the product of the elements in a vector.

Include: \verb+<vectlib.h>+

Format:

\begin{verbatim}
float  VfSum(   int n, float *  x );
double VdSum(   int n, double * x );
float  VfsSum(  int n, float *  x, int stride );
double VdsSum(  int n, double * x, int stride );

float  VfProd(  int n, float *  x );
double VdProd(  int n, double * x );
float  VfsProd( int n, float *  x, int stride );
double VdsProd( int n, double * x, int stride );
\end{verbatim}

Argument: {\tt n} -- the number of elements in the vector.

Argument: {\tt x} -- the vector of floating point numbers.

Argument: {\tt stride} -- the gap between successive elements of the vector.

Returns: the sum or the product of the elements of the vector.

Description: The Sum routines add together all the elements of the vector
and return the result. The Prod routines multiply together all the
elements of the vector. The vectors themselves remain unchanged. There
are variants for single and double precision and for whether or not
a stride is used.

For example, an implementation of the {\tt VdsSum()} routine in C might be:

\begin{verbatim}
double VdsSum( int n, double * x, int stride )
{ double result = 0.0;

  while (n-- > 0)
   { result += *x; x += stride; }

  return(result);
}
\end{verbatim}

See also: Vector Library, vector-vector operations, vector-scalar
operations, vector-scalar multiply, vector initialisation, vector
copying, vector dot products, vector maxima and minima
%%}}}
%%{{{  vector maxima and minima

\helpentry{vector maxima and minima}

Purpose: to determine the largest or smallest element of a vector.

Include: \verb+<vectlib.h>+

Format: 

\begin{verbatim}
int VfMax(   int n, float *  x );
int VdMax(   int n, double * x );
int VfsMax(  int n, float *  x, int stride );
int VdsMax(  int n, double * x, int stride );

int VfMin(   int n, float *  x );
int VdMin(   int n, double * x );
int VfsMin(  int n, float *  x, int stride );
int VdsMin(  int n, double * x, int stride );

int VfAmax(  int n, float *  x );
int VdAmax(  int n, double * x );
int VfsAmax( int n, float *  x, int stride );
int VdsAmax( int n, double * x, int stride );

int VfAmin(  int n, float *  x );
int VdAmin(  int n, double * x );
int VfsAmin( int n, float *  x, int stride );
int VdsAmin( int n, double * x, int stride );
\end{verbatim}

Argument: {\tt n} -- the number of elements in the vector.

Argument: {\tt x} -- the vector of numbers.

Argument: {\tt stride} -- the gap between successive elements of the vector.

Returns: an index within the vector for the desired element.

Description: these routines can be used to identify the largest or smallest
elements in a single vector. The Amax and Amin versions of the routine
ignore the signs of the vector elements and only consider the absolute
values. There are variants for single and double precision and for
whether or not a stride is used.

The number returned is an index within the vector. For example if a call
to {\tt VfsMax()} returns 0 then the first element of the vector is
the largest. If the same call returns $n - 1$ then the last element of the
vector is the largest.

For example, an implementation of the {\tt VdMin()} routine in C might be:

\begin{verbatim}
static int VdMin( int n, double * x )
{ int  index   = 0;
  double min     = *x;
  int  i;

  for (i = 1; i < n; i++)
   if (x[ i ] < min)
    { index = i; min = x[ i ]; }

  return(index);
}
\end{verbatim}

See also: Vector Library, vector-vector operations, vector-scalar
operations, vector-scalar multiply, vector initialisation, vector
copying, vector dot products, vector sums and products
%%}}}

\end{document}



