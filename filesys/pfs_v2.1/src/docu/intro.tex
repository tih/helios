\chapter{Introduction}

The Parsytec File System (PFS) is a powerful tool to handle mass storage 
devices in multi-transputer networks. Various applications which need highest 
data transfer rates can be realized in combination with SCSI devices which can
be accessed by Parsytec's ``Mass Storage Controller'' (MSC) board. Typical 
fields for applications which can make use of this product are for example 
image processing and distributed databases.

\section{Main features}

\begin{itemize}

  \item The Parsytec File System (PFS) is based on the Berkeley Fast File 
        System (FFS) up from version 4.2BSD. A lot of extensions have been 
        made to achieve maximal performance and to guarantee an optimal use of 
        the disc capacities.

  \item It is possible to connect up to seven SCSI devices to one MSC which
        then can be managed by one PFS. In addition to the standard devices, 
        new SCSI drives can easily be added by editing the driver's 
        \SIS\ configuration file.

  \item The PFS offers a concept of multivolume\slash multipartition. It is
        for example possible to declare two harddiscs to appear as one volume
        to the user (multipartition\index{multipartition}) or one harddisc may
        be split into several volumes (multivolume\index{multivolume}). It all
        depends on the volume and partition description of the user in \DIS.
        Devices with changeable media and raw devices are not allowed to be
        multipartitioned and have to be single volumes.

  \item The file system checker for structured volumes checks the consistency 
        of data and data structures of a file system. A lot of possible errors 
        are recovered; there are very few situations the checker cannot be 
        successfull. When the checker succeeds, the file system is guaranted to 
        be in a consistent state and the file server can work with it. There 
        are three checking modes: no checks, basic checks and full checks. The 
        checker is noninteractive and is called when a structured volume is 
        loaded (mounted).

  \item Several mechanisms have been added to handle fatal error situations. 
        Bad blocks of structured devices are reassigned automatically. When 
        buffer cache's checksum is enabled, data are only written on disc if 
        no inconsistence did occur. On error the data is left unchanged on 
        disc. Termination or unloading (unmounting) of volumes is done with 
        respect to active processes working on that volume.

  \item The basic block size is set to 4 KBytes (whereas the maximum file
        size\index{filesize@file size} is 4 GBytes). In dependancy on the
        client's request, the server is able to build packets of various sizes
        in its buffer cache to fit the request in an optimal way. The packet
        handling is done in a transparent manner to the client tasks.

  \item Directories and files are placed on the disc or partition by using 
        different strategies to get a good compromise between spreading and 
        clustering of information (minimal fragmentation). To implement such
        balancing procedures, the file server divides the disc into a number 
        of  equal sized ``cylinder groups'' which are controlled separately.

  \item In contrast to the BSD Fast File System the inode handling was 
        implemented in a more straight-forward way: inodes and entry names 
        build ``Directory Entries\index{directoryentries@directory entries}''
        which are kept in special directory blocks. You are able to create as
        many directories as you want, in one directory up to 448 entries
        (files or sub-directories) can be combined.

  \item To achieve maximal flexibility for the user, the server as well as the 
        driver is full parametrizable (via the \DI\ file, see section
        \ref{sec:di}, and the \SI\ file, see section \ref{sec:si}). With the
        desired hardware and application environment in mind, it is possible
        (and necessary) to manipulate the parameters of the devices and the
        buffer cache in a very wide range.

  \item Various utilities are added to the software package to ease handling 
        of the server. Especially protection mechanisms based on capabilities 
        are enabled.

  \item The {\tt ptar} utility helps you to archive lots of MBytes on your
        streamer tape.

  \item The PFS is delivered with a configurable device driver. This driver
        performs data transfer directly from and to the user's buffer, thereby
        eliminating intermediate copy operations. Requests regarding the same
        disc device are sorted to reduce head movement. By using the 
        transputer event line, reaction times could be minimised.

\end{itemize}

\section{The PFS data structures on disc}

This section gives you a brief description of important data structures as 
managed by the Parsytec File System. It will make easier to understand the 
basic principles of the file system checker (described below) and to follow 
the different phases of the checking process.

\subsection{Disc layout}

The Parsytec File System divides a disc into independant (logical) parts of 
equal sizes\footnote{They all contain the same number of blocks.}, called 
``cylinder groups\index{cylindergroups@cylinder groups}''. This approach has
several advantages: in combination with algorithms for block allocation and
creation of new directory entries and files, it can be assured that the
storage space on disc is used in the most efficient way:  on the one hand,
fragmentation of disc space is minimized, on the other hand, one gets optimal
medium access times by spreading data in a regular way over the disc.

All information required to manage a cylinder group is collected within one 
data block, known as an ``\/info block\index{infoblock@info block}''. Every
cylinder group has it's own info block and it is placed with a fixed
``relative rotational offset\index{rotationaloffset@rotational offset}''. The
most important data structure, building the major part of each info block, is
a ``bitmap''---an array of bytes used to note all free and allocated blocks
within the specific cylinder group. Some additional summary information give
a short description of the actual usage of a cylinder group. They contain e g
the number of free blocks and the number of sub-directories which are kept in 
a cyclinder group.

Another ``special purpose'' data block, called
``summary block\index{summaryblock@summary block}'', collects the summary
information from all cylinder group info blocks and some additional ones which
describe the whole file system with all cyclinder groups as one unit. Using
this summary information makes it especially easy and efficient for the file
server to decide where to allocate new blocks on disc to expand ane existing
file or to create a new sub-directory or file.

There should also be a place on disc where some fixed parameters of the
established file system can be found. These are for example items like the 
number of cylinder groups, the size of each cylinder group and the 
``rotational offset'' for a placement of info blocks within the different 
cylinder groups. The Parsytec File System puts this information and a lot of
other important data in a structure called ``superblock\index{superblock}''.
To have enough redundacy, a copy of this superblock is held within each info
block. Having formatted a disc once, the content of the superblock structure
should never be changed under normal operating conditions.

See figure \ref{pic:dlayout} for an overview.

\begin{figure}[htb]
  \small \sf
  \begin{figure}[htb]
  \small \sf
  \begin{figure}[htb]
  \small \sf
  \input{dlayout.pic}
  \caption{Disc layout overview}
  \label{pic:dlayout}
\end{figure}

  \caption{Disc layout overview}
  \label{pic:dlayout}
\end{figure}

  \caption{Disc layout overview}
  \label{pic:dlayout}
\end{figure}


\subsection{Directory entries and files}

All relevant information required to describe a sub-directory entry or a file
is kept in a data structure called ``\/inode\index{inode}''. These are for
example items like the creation date, the size (measured in bytes), and a type
flag to determine whether the entry describes a sub-directory, a symbolic
link, or an ordinary file. Additional information like an access matrix and
the number of blocks allocated by this entry and others can also be found
within in the corresponding inode.

To keep an inode as small as possible without loosing too much efficiency, 
only a restricted number of direct block references to data blocks are made
(``direct blocks\index{block!direct}''). If a larger file has to be managed,
a ``single indirect block\index{block!indirect!single}'' is allocated, which
exclusively contains references to data blocks. But one level of indirection
is not the limit: very large files of hundreds of Megabytes are described by
using a second level of indirection. In this case, a 
``double indirect block\index{block!indirect!double}'' is allocated and noted
in the inode which keeps references to single indirect blocks, which again
refer data blocks. This builds the basic mechanism to handle a wide range of
file sizes (up to 4 GBytes) without wasting too much disc space for allocating
blocks in advance.

There is one type of data block, called
``directory block\index{directoryblock@directory block}'' which is noted by an 
inode of type ``directory'' (a sub-directory entry) and represents a special 
variant of an ordinary data block: a directory block only contains inode data
structures and allows the Parsytec File System by this way to build
hierarchical directory structures.

\section{Checker}\index{checker|(}

The Parsytec File System Checker is a maintenance tool to guarantee 
consistency of block based file systems. Under various conditions, a file system
kept on a harddisc may come into an inconsistent state. Events like a power 
loss, a hardware malfunction or other disastrous errors like a head crash may
result in damaged data structures on disc. Under certain conditions, it may be 
even impossible to get initial access to the file system's root directory 
afterwards.

To recover from such an error and to repair corrupted parts of a file system the
Parsytec File System Checker was written. It builds an integral part of the 
Parsytec File System and is executed every time when a volume is loaded. By this
way, it can be assured in general that a file system is definitely in a 
consistent state after the checker has made all essential tests and corrections.

\subsection{How to use it \ldots}

The Parsytec File System Checker is started directly after the successful boot 
of a volume. All checks and corrections---if required---are done automatically
and there is no need for user interaction. In general, the file system checker 
works in a ``silent'' mode, only reporting the different phases of the 
checking processes which are entered. If an error occurs that needs 
corrections on disc, the checker reports it by writing a message to the server
window.

After finishing the checking process, the start-up sequence of the volume
is completed and the file system is accessable by the user in the normal 
manner\footnote{The checking process usually finishes successfully. Only under 
fatal error conditions like checking a non-formatted disc or working with a 
damaged disc controller, the File System Checker may fail. In this case, the 
file system cannot be booted, neither checked in the usual way, until the 
hardware malfunction or other reasons for the failure have been removed.}.

The system administrator has the opportunity to select from two different 
checking modes: at first, it is possible to perform a 
{\bf BASIC-CHECK}\index{basiccheck@BASIC-CHECK} when a volume is loaded. In this
case, all data structures on disc which are used to manage block allocation are
tested. If the File System Checker is called with 
{\bf FULL-CHECK}\index{fullcheck@FULL-CHECK} mode enabled, the complete 
directory tree is scanned and each entry is tested, regardless whether it is a 
file, a symbolic link or a sub-directory entry. This is the default checking 
mode. If some of the BASIC-CHECKs fail, the FULL-CHECK mode is called 
automatically to detect other possible inconsistencies and to correct them 
afterwards.

The Parsytec File System can be booted with various options. The checker 
relevant ones are

\begin{tabular}{ll}
  -f: & perform FULL-CHECKs (default) \\
  -b: & perform only BASIC-CHECKs     \\
  -n: & bypass the checker completely \\
\end{tabular}

There may be one fatal error condition which makes it impossible to run the 
checker and to make corrections properly: a damaged superblock within the 
info block of the first cylinder group. In this case, differences between the 
data structures describing a physical disc and data kept within \DI\ may be
recognized. Run the {\tt mksuper} command to make the file system bootable
again\footnote{An empty file system is created with {\tt makefs}, that
destroys all data when executed on a file system that already contains
files.}. All required information is taken from the actual \DI\ file and a new
superblock structure is filled up. Afterwards, the file server can be booted
in the normal manner.

\subsection{The different phases of the checking process}

Consistency checking and correction operations done by the Parsytec File 
System Checker are split into four main phases:


\subsubsection{Phase I}
contains all tests which are executed running in
BASIC-CHECK\index{basiccheck@BASIC-CHECK} mode:

At first, the validity of the superblock data structure as found in the 
info block of the first cylinder group is prooved. Afterwards, this superblock 
is compared with the copies found in the info blocks of the other cylinder 
groups. The next step is to compare the content of the different cylinder 
group bitmaps with the summary information kept in the info blocks and in the
summary block.

Finally, the root directory inode is inspected to make sure that it is 
possible to get initial access to the file system's root directory. (The root 
directory inode is part of the summary block.)

If the Parsytec File System Checker is called in BASIC-CHECK mode and no error 
was detected, the checking process is finished at this moment. Otherwise, the 
FULL-CHECK mode is automatically entered and the following three phases are 
also performed.

\subsubsection{Phase II}
is used to traverse the complete directory tree and to make all essential 
checks for single directory entries. These test operations build the main part 
of the checking process when running in FULL-CHECK mode: beginning with the 
inode of the root directory, the whole directory tree is worked out and each 
individual entry is tested under various conditions. If certain error limits are
exceeded, an entry may be detected as ``non-usable'' and is deleted afterwards.

\subsubsection{Phase III}
takes the results of the Phases I and II and corrects all block allocation
errors which were previously detected:

There are two serious error conditions which are handled during phase III. At 
first, there may be a block which is referred by more than one entry. The file
system checker has to decide which entry may be the legal owner (most recent
modifier) of such a block and has to do the required corrections. Secondly, 
there may be a block which is noted in a bit-map as allocated but not referred 
by any of the entries touched during phase II. Such a ``lost'' block may become 
an entry in the 
{\tt /lost+found}\index{lostandfound@{\tt /lost+found}}-directory if it contains
directory information (inodes). By this way, it is possible to re-integrate
sub-directory structures which were cut-off due to an error condition.

\subsubsection{Phase IV}
performs some general tidyup-operations and adjusts the file system's summary 
information which are kept in the summary block.\index{checker|)}

\section{Related manuals}
The following manuals may be useful as references:

\begin{itemize}

  \item MSC Mass Storage Controller  Technical Documentation

  \item Perihelion Technical Report No. 20  Device Configuration

  \item The Helios Parallel Operating System

  \item ANSI X2.131-1986 Small Computer System Interface

  \item Manuals to your SCSI devices

\end{itemize}

In addition to that, the file {\tt readme.pfs} contains most actual changes not
considered in this handbook.
