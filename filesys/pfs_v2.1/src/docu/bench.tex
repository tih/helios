\chapter{Benchmarks} \label{chap:benchmarks}

To fit you for both, writing your own programs using the PFS and testing the
data transfer rate of your system, we implemented two small C-programs and
some shell scripts to work with them. 

\section{The programs}

\begin{description}

  \item {\tt create}\index{create@{\tt create}} \\
        can be used to generate a file of a given size. We used it to create
        {\tt kbyte.1} (of size 1 KByte), {\tt block.1} (of size 4 KBytes) and
        {\tt packet.1} (of size 64 KBytes). In general, it's called like

        \fbox {\tt \% create <ContextDir> <FileName> <FileSize>}

        to create the file {\tt <ContextDir>/<FileName>} that will contain
        {\tt <FileSize>} bytes. See {\tt copy.c} for the sources.

  \item {\tt copy}\index{copy@{\tt copy}} \\
        can be used to measure PFS' data transfer rate. It's called

        \fbox{
          \vbox{
            \tt \% copy <PathToSrc> <PathToDest> <Path\-To\-Volume>
            <Read\-Count> <Write\-Count>
          }
        }

        to copy the file {\tt <PathToSrc>} to {\tt <PathToDest>} using the
        file server for {\tt <Path\-To\-Volume>}. If you want to measure the
        transfer rate from/to your host you may give values to
        {\tt <ReadCount>} and {\tt <WriteCount>} which differ from one. So you
        can simulate a 20 MByte file on your PC's harddisc that claims only 1
        MByte by reading\slash writing it twenty times. See {\tt copy.c},
        {\tt csync.c} and {\tt pathsplt.c} for the sources.

  \item {\tt cppfs}\index{cppfs@{\tt cppfs}} \\
        is a shell script pattern to ease the usage of {\tt copy}. It has to
        be modified to meet your environment.

  \item {\tt demo1}-{\tt demo4}\index{demo@{{\tt demo1}-{\tt demo4}}} \\
        conduct some standard tests.

\end{description}
