\section{fs}\index{fs@{\tt fs}}
\begin{man}
  \PP Start the Parsytec File Server.
  \FO {\tt fs [-c][-o][-f|-b|-n] [<DevInfoName>] <FileServer>}
  \DE The file server is started, loads {\tt <DevInfoName>} (when given), tries
             to install the file server specified in the {\tt <FileServer>} (usually of 
             value `{\tt msc21}') block. The file server has to be started on a MSC in the
             background. This is normally done with

             \bigskip
             \fbox{\tt remote -d MSC fs msc21}
             \bigskip

             Options:
             \begin{itemize}

             \item {\tt -c}\newline
               The use of the buffer cache checksum is enabled. Using this mode
               reduces speed of file server by factor three.
	       Default is working without buffer cache checksum.

             \item {\tt -o}\newline
               If this option is given the file server reports open requests
               to the server window.

             \item {\tt -f}, {\tt -b}, {\tt -n}\newline
               These options work with structured volumes only.
	       The checking mode is determined. This option is overridden by 
               the checking mode option of a {\tt load} command. Default checking
               mode is {\tt -f}.

               \begin{itemize}

               \item {\tt -f}\newline
                 Full checks; file system data and directory trees are checked.

               \item {\tt -b}\newline
                 Basic checks; file system data is checked and on occurrence of
	         errors directory trees are checked.

               \item {\tt -n}\newline
                 No checks; checker is bypassed completely.

               \end{itemize}
             \end{itemize}

             \begin{note}
               {\tt fs} allocates memory for all volumes specified in \DI. So you
               have to execute a {\tt termvol} command for those volumes to terminate
               the file server and to clean the memory, even if some volumes have
               not been loaded.
             \end{note}

  \SA {\tt load}, {\tt unload}, {\tt termvol}
\end{man}
