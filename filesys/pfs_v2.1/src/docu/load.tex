\section{load}\index{load@{\tt load}}
\begin{man}
  \PP Load a volume.
  \FO {\tt load [-v][-l][-m][-f|-b|-n] <PathToVolume>}
  \DE The volume specified by {\tt <PathToVolume>} is loaded (mounted)
             and for structured volumes the checker is called and a file system
             is tried to be set up.
             After file server startup this has to be done with every volume
             explicitly.
             Changeable media are locked after they have been loaded so that
             they cannot be removed until an unload command is given.

             Options:
             \begin{itemize}

             \item {\tt -v}\newline
               The {\tt load} command waits for the completion of the load and
               reports about the results. On success {\tt load} reports about the
               number of cylinder groups and blocks per cylinder group of the
               loaded file system. This option has no effect if the {\tt -m} option
               is given. Default is not to wait for completion.

             \item {\tt -m}\newline
               This option works with structured volumes only.
	       The specific volume is loaded but the checker is not called and
	       no file system is tried to be set up. This option must be applied
	       before using the {\tt makefs}, {\tt format} or {\tt mksuper} command.
	       This option disables the {\tt -v} option.

             \item {\tt -f}, {\tt -b}, {\tt -n}\newline
               These options work with structured volumes only.
	       The checking mode is determined. If none of these options is
	       given the checking mode determined in the file server commandline
	       is used. If no specific checking mode was given there the 
               default mode ({\tt -f}) is used.

               \begin{itemize}
               \item {\tt -f}\newline
                 Full checks; file system data and directory trees are checked.

               \item {\tt -b}\newline
                 Basic checks; file system data is checked and on occurrence of
	         errors directory trees are checked.

               \item {\tt -n}\newline
                 No checks; checker is bypassed completely.
               \end{itemize}

             \item {\tt -l}\newline
               This option only has an effect during a full check.
	       If there are `hanging' symbolic links detected after a full
               check these links will be destroyed. Default is not to destroy
               `hanging' links.
             \end{itemize}

  \SA {\tt unload}, {\tt termvol}
\end{man}
