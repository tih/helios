\section{ptar}\index{ptar@{\tt ptar}}
\begin{man}
  \PP Store files in an archive.
  \FO {\tt ptar <Options> <Files>}
  \DE {\tt ptar} allows you storing copies of files in an archive.

             Options:

             \begin{itemize}
             \item {\tt -c}, {\tt -d}, {\tt -t}, {\tt -x}\newline
               These four option switch between the operation modes.

               \begin{itemize}
               \item {\tt -c}\newline
                 Create a new archive.

               \item {\tt -d}\newline
                 Compare the files in the archive with those in the file system
                 and report about differences.

               \item {\tt -t}\newline
                 Display a list a the files in the archive.

               \item {\tt -x}\newline
                 Extract files from archive.
               \end{itemize}

             \item {\tt -B}, {\tt -C}, {\tt -f}, {\tt -M}, {\tt -N}, {\tt -R}, {\tt -T}, {\tt -v}, {\tt -w}\newline
               General options.

               \begin{itemize}
               \item {\tt -B <Number>}\newline
                 Set blocking factor to {\tt <Number>}.

               \item {\tt -C <Directory>}\newline
                 Change into {\tt <Dirtectory>} before continuing.

               \item {\tt -f <Filename>}\newline
                 Archive files in {\tt <Filename>} (instead of using the value of
                 {\tt TARFILE} respectively `{\tt tar.out}').

               \item {\tt -M}\newline
                 Work on a multi-volume archive.

               \item {\tt -N <Date>}\newline
                 Work only on files whose creation or modification date is
                 newer than {\tt <Date>}.

               \item {\tt -R}\newline
                 Print each message's record number.

               \item {\tt -T <Filename>}\newline
                 Work on the list of files in {\tt <Filename>}, too.

               \item {\tt -v}\newline
                 Enter verbose mode.

               \item {\tt -w}\newline
                 Wait for user's confirmation before every action.
               \end{itemize}
             \end{itemize}
\end{man}
             \begin{itemize}
             \item {\tt -h}, {\tt -V}\newline
               Creation options.

               \begin{itemize}
               \item {\tt -h}\newline
                 Treat simbolic links as normal files or directories.

               \item {\tt -V <Name>}\newline
                 Write a volume header at the beginning of the archive.
               \end{itemize}

             \item {\tt -k}, {\tt -m}, {\tt -p}\newline
               Extraction option:

               \begin{itemize}
               \item {\tt -k}\newline
                 Keep existing files in the file system.

               \item {\tt -m}\newline
                 Do not extract the modification and access date from archive.

               \item {\tt -p}\newline
                 Set access matrices as recorded in the archive.
               \end{itemize}
             \end{itemize}

             See chapter ``Backups'' for detailed information.
\newpage