\chapter{Upgrade}\label{chap:upgrade}\index{upgrade|(}\index{update|(}

This chapter covers the hardware changes and software upgrade to use PFS v2.1.
It has only to be read by former users of an old MSC board, the Helios File
System v1.x, and the Parsytec File System v2.0.x. You will find a step by step
guideline for a proper update to the PFS v2.1.

\section{Backups}
If you follow the instructions in this chapter, you should find a perfectly 
working PFS afterwards. But:
\begin{verse}\index{Murphy}
{\large \sf If anything can go wrong, it will.} \\
\end{verse}

With this in mind, it's urgently recommended to backup your data to your host
system before installing the new PFS.

\section{Hardware update}  \label{sec:hwupdate}
This section has to be read by all users performing an update.

There are several requirements for proper operation of the MSC Device Driver
(Jumper and EPROM placement is shown in figure \ref{pic:msc}):

\begin{itemize}

  \item Xilinx EPROM\index{xinlinxeprom@Xilinx EPROM}\index{EPROM} \\
        The EPROM for Xilinx initialisation has to be updated to Revision 6
        (at least). Only these revisions support event generation as necessary
        for the driver. The EPROM has to be placed as shown in figure
        \ref{pic:eprom}. Please check which version is installed on your MSC.

        \begin{figure}[hbt]
  \small \sf
  \begin{figure}[hbt]
  \small \sf
  \begin{figure}[hbt]
  \small \sf
  \input{eprom.pic}
  \caption{Xilinx EPROM}
  \label{pic:eprom}
\end{figure}

  \caption{Xilinx EPROM}
  \label{pic:eprom}
\end{figure}

  \caption{Xilinx EPROM}
  \label{pic:eprom}
\end{figure}


  \item Jumper\index{jumper} settings \\
        Some jumpers have to be adjusted to ensure suitable event generation:

        \begin{itemize}

          \item The ERROR PAL (U52) activates the ERRORIN\index{ERRORIN} input
                of the transputer when the data transfer controller or the
                extension board requests an event. J0 enables or disables the
                activation of the ERRORIN input in by the error PAL cause of 
                an error condition: program error, address error or parity 
                error. It has to be placed in 1-2 position as shown in figure
                \ref{pic:jumper0}.

                \begin{figure}[hbt]
  \small \sf
  \begin{figure}[hbt]
  \small \sf
  \begin{figure}[hbt]
  \small \sf
  \input{jumper0.pic}
  \caption{Jumper J0}
  \label{pic:jumper0}
\end{figure}

  \caption{Jumper J0}
  \label{pic:jumper0}
\end{figure}

  \caption{Jumper J0}
  \label{pic:jumper0}
\end{figure}


          \item J9 makes connection between the ERRORIN pin and the
                EVENTREQ\index{EVENTREQ} pin of the transputer. So if this
                jumper is inserted as shown in figure \ref{pic:jumper9} an
                activation of ERRORIN of the ERROR PAL U52 also generates an
                event.

                \begin{figure}[hbt]
  \small \sf
  \begin{figure}[hbt]
  \small \sf
  \begin{figure}[hbt]
  \small \sf
  \input{jumper9.pic}
  \caption{Jumper J9}
  \label{pic:jumper9}
\end{figure}

  \caption{Jumper J9}
  \label{pic:jumper9}
\end{figure}

  \caption{Jumper J9}
  \label{pic:jumper9}
\end{figure}


        \end{itemize}

\end{itemize}

Figure \ref{pic:msc} shows the exact location of the EPROM and the two jumpers.

\begin{figure}[p]
  \small \sf
  \vspace{18cm}
  \caption{MSC board}
  \label{pic:msc}
\end{figure}


If these requirements are not met, the device driver will not start up.
Instead it will send a message to the server window, complaining about 
missing events from the hardware.

It's not too difficult to carry out the necessary changes on your own, but if
you are thumb-fingered or want to prevent loss of warranty of your MSC board,
please send the board to Parsytec. We will perform the update (including
tests) for you.


\section{Software upgrade}

The general installation procedure is described in chapter
\ref{chap:install}. Refer to that and install the PFS before you proceed. The
next sections go into detail about changes of your \DIS. In addition to that,
the declaration of your MSC in your resource map has changed. For more
details see section \ref{sec:runpfs}.

\subsection{\DI\  file, coming from PFS v2.0.x}
\index{devinfo@\DI}\index{pfs20x@PFS v2.0.x}
This section is only for users who have worked with PFS v2.0.x up to now. It
is assumed that you have worked through the device configuration example in
the PFS v2.0 manual. Therefore, only the differences to the original
{\tt bible.src} file are outlined here.

For each drive, the {\it type} and {\it id} specifications have to be changed.
The {\it type} entry now refers to the position of the device declaration in
the \SIS\ file (which is the configuration file for the MSC device driver, see
chapter \ref{chap:msc}), and {\it id} values are now given as hexadecimal
numbers to allow the specification of different Logical Unit Numbers within a
single device (0x31 refers to SCSI address 3, LUN 1).

The entry for the Wren VI drive will now look as follows:

\begin{listing}
  \begin{verbatim}
drive        # Wren VI
{
  id   0x30  # SCSI address 3, LUN 0
  type 0     # first entry in scsiinfo
}
  \end{verbatim}
\end{listing}

The description of partitions has not been changed, but the {\it controller}
specification of the {\it discdevice} block. It's now given in hexadecimal
notation, too. Thus, your {\it discdevice} entry for the MSC should be:

\begin{listing}
  \begin{verbatim}
discdevice msc21          # MSC driver for PFS v2.1
{
  name        msc21.dev   # refers to /helios/lib/msc21.dev
  controller  0x70        # uses Address 7, LUN 0
  addressing  1024        # use 1 KByte blocks

  partition               # number 0
  {
    :
  }

  :
  :

  drive                   # number 0
  {
    :
  }
}                         # end of discdevice
  \end{verbatim}
\end{listing}

After modifying the \DIS\ file, compile it to binary form with

\fbox{{\tt \% gdi} \DIS\ \HEDI}

\subsection{\DI\ file, coming from HFS v1.x}
\index{devinfo@\DI}\index{hfs1x@HFS v1.x}
This section is only for users who have worked with HFS v1.x up to now.

\begin{caution}
  This is a mini example for those users who work with only one MSC connected
  to one harddisc.
\end{caution}

\begin{enumerate}

  \item Create a new file {\tt upgrade.src}\index{upgradesrc@{\tt upgrade.src}}
        in the directory \HE\ (based on {\tt /helios\slash info.src\slash
        hfs2pfs.src}):

\begin{listing}
  \begin{verbatim}
fileserver msc21
{
  device       msc21
  blocksize    4096

  smallpkt     1
  mediumpkt    4
  hugepkt      16

  smallcount   20
  mediumcount  8
  hugecount    28

  volume
  {
    name      <Old>  # The file server doesn't care about your
                     # volume's name and you could change it any
                     # time, but with respect to consistency of
                     # shell scripts, makefiles, and user programs
                     # it's recommended to use the old name
    partition 0
    cgsize    <Old>  # This value MUST be taken from the old
                     # devinfo.src, as well as
    ncg       <Old>  # this one.
  }
}

discdevice msc21
{
  name        msc21.dev
  controller  0x70
  addressing  1024

  partition
  {
    drive 0
  }

  drive
  {
    id    0x00  # Usually, only one harddisc is connected to the
                # MSC and occupies SCSI address 0 (1st nibble),
                # Logical Unit Number 0 (second nibble).
    type  0     # This refers to a device description number in
                # the scsiinfo.src file. Probably you have a WREN
                # harddisc, it's description is at position number 0.
  }
}
  \end{verbatim}
\end{listing}

  \item Translation: \\

          \fbox{\tt \% gdi /helios/etc/upgrade.src /helios/etc/upgrade.di}

  \item Starting the file server, loading the only volume (prepared for
        {\tt mksuper}): \\

          {\tt
            \begin{tabular}{|l|}
              \hline
              \% remote -d MSC fs upgrade.di msc21 \\
              \% load -m /<VolumeName>             \\
              \hline
            \end{tabular}
          }

  \item  Now, the file server tries to find out the correct disc parameters,
         following this description (decreasing priority):

         \begin{enumerate}

           \item Reading the superblock from the harddisc \\
                 The HFS has stored the values for {\it cgsize} and {\it ncg} in
                 this block.

           \item Evaluating the configuration file, \\
                 which is \HE{\tt\slash upgrade.di} in this case. If the
                 devinfo-parameter was not given at file server's startup
                 time, \HEDI\ is used by default.

                 \begin{caution}
                   If the superblock is damaged, the priority is given to the
                   values in the configuration file. For that reason,
                   {\it cgsize} and {\it ncg} in {\tt upgrade.src} {\bf must}
                   be given the same values as in your old \DIS.
                 \end{caution}

         \end{enumerate}

  \item Write parameters to disc \\
        Now, the correct PFS parameters have to be written to the superblock
        by executing:

        \fbox{\tt \% mksuper /<VolumeName>}

        In some cases there may be problems: you have discovered the values 
        for {\it cgsize} and {\it ncg} and have written them down to \DI, but
        the {\tt mksuper} command couldn't be executed successfully. This
        behaviour occurs if the product of {\it ncg} and {\it cgsize} exceeds
        the size of your harddisc. This error was not checked by the HFS and
        the wrong parameters were written to disc. To solve this problem,
        decrease {\it ncg} in {\tt upgrade.src}, compile it, and restart the
        file server until {\tt mksuper} is performed succesfully. {\bf By
        doing this, some data may get lost, of course.} So {\bf do} backup
        your file system now at the latest (use your old \DI) to rescue as
        many data as possible.

\end{enumerate}
\index{upgrade|)}\index{update|)}
