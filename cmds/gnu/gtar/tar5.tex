% -*-LaTeX-*-
% Converted automatically from troff to LaTeX by tr2tex on Wed Jan 31 09:46:58 1990
% tr2tex was written by Kamal Al-Yahya at Stanford University
% (Kamal%Hanauma@SU-SCORE.ARPA)


\documentstyle[troffman]{article}
\begin{document}
%
% input file: tar.5
%
\phead{TAR}{5}{15\ October\ 1987}

% @(#)tar.5 1.4 11/6/87 Public Domain - gnu
\shead{NAME}
tar -- tape (or other media) archive file format
\shead{DESCRIPTION}
A ``tar tape'' or file contains a series of records.  Each record contains
TRECORDSIZE bytes (see below).  Although this format may be thought of as
being on magnetic tape, other media are often used.
Each file archived is represented by a header record
which describes the file, followed by zero or more records which give the
contents of the file.  At the end of the archive file there may be a record
filled with binary zeros as an end-of-file indicator.  A reasonable
system should write a record of zeros at the end, but must not assume that
an end-of-file record exists when reading an archive.

The records may be blocked for physical I/O operations.  Each block of
%
\it N \rm%
 records (where %
\it N \rm%
 is set by the %
\bf --b \rm%
 option to %
\it tar\rm%
)
is written with a single write() operation.  On
magnetic tapes, the result of such a write is a single tape record.
When writing an archive, the last block of records should be written
at the full size, with records after the zero record containing
all zeroes.  When reading an archive, a reasonable system should
properly handle an archive whose last block is shorter than the rest, or
which contains garbage records after a zero record.

The header record is defined in the header file $<$tar.h$>$ as follows:
\par\vspace{0.5ex}
\begin{verbatim}
/*
 * Standard Archive Format - Standard TAR - USTAR
 */
#define RECORDSIZE    512
#define NAMSIZ        100
#define TUNMLEN       32
#define TGNMLEN       32

union record {
        char    charptr[RECORDSIZE];
        struct header {
                char    name[NAMSIZ];
                char    mode[8];
                char    uid[8];
                char    gid[8];
                char    size[12];
                char    mtime[12];
                char    chksum[8];
                char    linkflag;
                char    linkname[NAMSIZ];
                char    magic[8];
                char    uname[TUNMLEN];
                char    gname[TGNMLEN];
                char    devmajor[8];
                char    devminor[8];
        } header;
};

/* The checksum field is filled with this while the checksum is computed. */
#define CHKBLANKS "        "        /* 8 blanks, no null */

/* The magic field is filled with this if uname and gname are valid. */
#define TMAGI     "ustar  "         /* 7 chars and a null */

/* The linkflag defines the type of file */
#define LF_OLDNORMAL '\0'                 /* Normal disk file, Unix compatible */
#define LF_NORMAL    '0'                  /* Normal disk file */
#define LF_LINK      '1'                  /* Link to previously dumped file */
#define LF_SYMLINK   '2'                  /* Symbolic link */
#define LF_CHR       '3'                  /* Character special file */
#define LF_BLK       '4'                  /* Block special file */
#define LF_DIR       '5'                  /* Directory */
#define LF_FIFO      '6'                  /* FIFO special file */
#define LF_CONTIG    '7'                  /* Contiguous file */
/* Further link types may be defined later. */

/* Bits used in the mode field - values in octal */
#define TSUID           04000             /* Set UID on execution */
#define TSGID           02000             /* Set GID on execution */
#define TSVTX           01000             /* Save text (sticky bit) */

/* File permissions */
#define TUREAD  00400                     /* read by owner */
#define TUWRITE 00200                     /* write by owner */
#define TUEXEC  00100                     /* execute/search by owner */
#define TGREAD  00040                     /* read by group */
#define TGWRITE 00020                     /* write by group */
#define TGEXEC  00010                     /* execute/search by group */
#define TOREAD  00004                     /* read by other */
#define TOWRIT  00002                     /* write by other */
#define TOEXEC  00001                     /* execute/search by other */
\end{verbatim}
\par\noindent
All characters in header records
are represented using 8-bit characters in the local
variant of ASCII.
Each field within the structure is contiguous; that is, there is
no padding used within the structure.  Each character on the archive medium
is stored contiguously.

Bytes representing the contents of files (after the header record
of each file) are not translated in any way and
are not constrained to represent characters or to be in any character set.
The %
\it tar\rm%
(5) format does not distinguish text files from binary
files, and no translation of file contents should be performed.

The fields %
\it name, linkname, magic, uname\rm%
, and %
\it gname \rm%
 are
null-terminated
character strings.  All other fields are zero-filled octal numbers in
ASCII.  Each numeric field (of width %
\it w\rm%
) contains %
\it w\rm%
-2 digits, a space, and
a null, except %
\it size \rm%
 and %
\it mtime\rm%
,
which do not contain the trailing null.

The %
\it name \rm%
 field is the pathname of the file, with directory names
(if any) preceding the file name, separated by slashes.

The %
\it mode \rm%
 field provides nine bits specifying file permissions and three
bits to specify the Set UID, Set GID and Save Text (TSVTX) modes.  Values
for these bits are defined above.  When special permissions are required
to create a file with a given mode, and the user restoring files from the
archive does not hold such permissions, the mode bit(s) specifying those
special permissions are ignored.  Modes which are not supported by the
operating system restoring files from the archive will be ignored.
Unsupported modes should be faked up when creating an archive; e.g.
the group permission could be copied from the `other' permission.

The %
\it uid \rm%
 and %
\it gid \rm%
 fields are the user and group ID of the file owners,
respectively.

The %
\it size \rm%
 field is the size of the file in bytes; linked files are archived
with this field specified as zero.

The %
\it mtime \rm%
 field is the modification time of the file at the time it was
archived.  It is the ASCII representation of the octal value of the
last time the file was modified, represented as in integer number of
seconds since January 1, 1970, 00:00 Coordinated Universal Time.

The %
\it chksum \rm%
 field is the ASCII representaion of the octal value of the
simple sum of all bytes in the header record.  Each 8-bit byte in the
header is treated as an unsigned value.  These values are added to an
unsigned integer, initialized to zero, the precision of which shall be no
less than seventeen bits.  When calculating the checksum, the %
\it chksum \rm field is treated as if it were all blanks.

The %
\it typeflag \rm%
 field specifies the type of file archived.  If a particular
implementation does not recognize or permit the specified type, the file
will be extracted as if it were a regular file.  As this action occurs,
%
\it tar \rm%
 issues a warning to the standard error.
\begin{description}
\item[{LF\_NORMAL\ or\ LF\_OLDNORMAL}]
represents a regular file.
For backward compatibility, a %
\it typeflag \rm%
 value of LF\_OLDNORMAL
should be silently recognized as a regular file.  New archives should
be created using LF\_NORMAL.
Also, for backward
compatability, %
\it tar \rm%
 treats a regular file whose name ends
with a slash as a directory.
\item[{LF\_LINK}]
represents a file linked to another file, of any type,
previously archived.  Such files are identified in Unix by each file
having the same device and inode number.  The linked-to
name is specified in the %
\it linkname \rm%
 field with a trailing null.
\item[{LF\_SYMLINK}]
represents a symbolic link to another file.  The linked-to
name is specified in the %
\it linkname \rm%
 field with a trailing null.
\item[{LF\_CHR\ or\ LF\_BLK}]
represent character special files and block
special files respectively.
In this case the %
\it devmajor \rm%
 and %
\it devminor \rm fields will contain the
major and minor device numbers respectively.  Operating
systems may map the device specifications to their own local
specification, or may ignore the entry.
\item[{LF\_DIR}]
specifies a directory or sub-directory.  The directory name
in the %
\it name \rm%
 field should end with a slash.
On systems where
disk allocation is performed on a directory basis the %
\it size \rm field will contain the maximum number of bytes (which may be
rounded to the nearest disk block allocation unit) which the
directory may hold.  A %
\it size \rm%
 field of zero indicates no such
limiting.  Systems which do not support limiting in this
manner should ignore the %
\it size \rm%
 field.
\item[{LF\_FIFO}]
specifies a FIFO special file.  Note that the archiving of
a FIFO file archives the existence of this file and not its
contents.
\item[{LF\_CONTIG}]
specifies a contiguous file, which is the same as a normal
file except that, in operating systems which support it,
all its space is allocated contiguously on the disk.  Operating
systems which do not allow contiguous allocation should silently treat
this type as a normal file.
\item[{`A'\ -\ `Z'}]
are reserved for custom implementations.  None are used by this
version of the %
\it tar \rm%
 program.
\item[{\it other\rm}]
values are reserved for specification in future revisions of the
P1003 standard, and should not be used by any %
\it tar \rm%
 program.
\end{description}
The %
\it magic \rm%
 field indicates that this archive was output in the P1003
archive format.  If this field contains TMAGIC, then the
%
\it uname \rm%
 and %
\it gname \rm%
fields will contain the ASCII representation of the owner and group of the
file respectively.  If found, the user and group ID represented by these
names
will be used rather than the values contained
within the %
\it uid \rm%
 and %
\it gid \rm%
 fields.
User names longer than TUNMLEN-1 or group
names longer than TGNMLEN-1 characters will be truncated.
\shead{SEE ALSO}
tar(1), ar(5), cpio(5), dump(8), restor(8), restore(8)
\shead{BUGS}
Names or link names longer than NAMSIZ-1 characters cannot be archived.

This format does not yet address multi-volume archives.
\shead{NOTES}
This manual page was adapted by John Gilmore
from Draft 6 of the P1003 specification
\end{document}
